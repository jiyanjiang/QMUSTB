\chapter*{写在前面的话}

\qquad ``量子力学''已经诞生近百年,量子力学的应用范围也早已超越了``原子物理''
、``核物理''以及``粒子物理''的范畴,在``凝聚态物理''、``生物物理''、
``电子学''等广泛领域都有重要的应用,可以说``量子力学''已成为科学家的通用语言,
其重要性不言而喻。

量子力学是以原子物理的研究发展而来的,所以导论性量子力学课程与原子物理的关联很密切。
大多数情形下我们是通过对原子物理的学习来引入量子力学中的关键概念,量子力学理论框架建立后,
往往我们也是针对原子物理中的问题来应用量子力学。
原则上我们可将导论性的量子力学和原子物理看作一个整体来讲授和学习。

在本课程的前半部分主要讨论概念,并建立量子力学的基本理论,理解和思辩的成分较多,
类似于普通物理的学习。后半部分主要讨论量子力学的应用和近似方法,会涉及较多数学,
推导较多,概念相对较少,即便有也是基于数学基础上的概念,
学习的过程属于典型的理论物理课程。

不要期望一个学期就可掌握量子力学,对物理专业的学生来说量子力学是个反复学习的过程。
一般而言可分为四个阶段:(1)导论性的量子力学;
(2)高等量子力学,重视量子力学的形式体系,但一般不包括相对论性量子力学;
(3)量子场论,包括相对论性量子力学;(4)更高级的或更专门的量子场论。

本讲义适用于物理学专业或应用物理学专业的导论性量子力学,
对量子力学感兴趣的非物理专业学生也可参考。


关于``量子力学''国内外已有相当多的经典著作,但随着科学技术的不断进步,
特别是随着``量子力学''本身不断发展并应用到不同物理系统中,
我们的教材/讲义必须能反映这种最新进展。在内容的组织上,本讲义以``量子力学''为主线,
并在概念的引入,计算方法的应用中贯穿``原子物理''的知识,
并同时介绍``量子力学''的最新进展及其应用。

常用参考书:

\begin{enumerate}

\item{周世勋:《量子力学教程》,高等教育出版社;内容精练。}

\item{David J.Griffiths, \textbf{Introduction to Quantum Mechanics}, 机械工业出版社(2006).
美国本科生用标准教材,中英文版国内均有售。}

\item{J. J. Sakurai, \textbf{Modern Quantum Mechanics}, Revised Edition,
世界图书出版公司(2006). 美国研究生用标准教材,英文版国内有售。}
\end{enumerate}


更多参考书:


\begin{enumerate}


\item{曾谨言:《量子力学》卷I,科学出版社;内容深入浅出,讲解详细。}

\item{曾谨言:《量子力学导论》,北京大学出版社;适合本科生初学使用。}

\item{苏汝铿:《量子力学》,高等教育出版社;内容丰富,包括不少量子力学的新进展。}

\item{顾莱纳:《量子力学导论》,北京大学出版社;讲解详细,
是作者理论物理系列教材中的一本。}

\item{David Bohm, {\bf Quantum Theory}, Dover Publications (1989).
美国研究生用标准教材,有中译本。}

\item{Ramamurti Shankar, {\bf Principles of Quantum Mechanics}, Plenum Press, (1994).}

\item{P. A. M. Dirac, {\bf The Principles of Quantum Mechanics},
Clarendon Press, (1958). 量子力学经典著作,有中译本。}

\item{钱伯初、曾谨言:《量子力学习题精选与剖析(上)》,
科学出版社。}

\item{张宏宝:《量子力学教程学习辅导书》, 高等教育出版社。与周世勋书配合使用。}

\item{扬福家:《原子物理学》,高等教育出版社;有关于量子力学发展的大量背景资料和对相关物理概念的讨论,适合阅读。}

\item{杨桂林 等:《近代物理》,科学出版社(2004)}

\item{一本数学手册。}

\item{量子力学的最新进展与应用:
\\科学杂志新闻(Science News),
\url{http://news.sciencemag.org/}
\\自然杂志新闻(Nature News),
\url{http://www.nature.com/news/index.html}
\\科学的美国人(Scientific American Magazine),
\url{http://www.sciam.com/} 有中文版。
\\今日物理(Physics Today),
\url{http://www.physicstoday.org/}
\\物理世界(Physics World),
\url{http://physicsworld.com}

%\\物理评论聚焦(Physical Review Focus),
%\url{http://focus.aps.org/} 

}

\item{更专门的物理论文网站:
\\美国物理学会(APS)期刊的焦点论文,
\url{http://physics.aps.org/}
\\开放获取的物理论文网站(arXiv),
\url{http://cn.arxiv.org/} 
\\New Journal of Physics,
\url{http://iopscience.iop.org/1367-2630}

}

\end{enumerate}


\bigskip\medskip

\noindent
@季燕江\\

Email: jyj@sas.ustb.edu.cn\\
Git: \url{https://github.com/jiyanjiang/QMUSTB}\\
豆瓣: \url{http://site.douban.com/223228/}\\

20010年6月