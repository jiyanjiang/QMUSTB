\section{二次量子化}

\begin{quotation}
“类比说的不是相似,而是关系。”\qquad 麦克斯韦
\end{quotation}

对全同多粒子系统而言,我们可以使用直接乘积表象,但这样得到的波函数是很啰唆的,而且必须额外保证这样得到的波函数是对称化(玻色子)或反对称化(费米子)形式的。由于我们无法说出哪一个宗量$q_i$对应的量子态是哪一个$n$,所以我们只需说多粒子系统中有几个粒子处在$m$态,有几个粒子处在$n$态即可,这就是所谓占有数表象。

\subsection{产生、湮灭算符}

\subsubsection{费米系统}

对多费米子系而言,假设最简单的情况两个费米子\index{Fermion: 费米子},一个费米子处在$m$态,另一个费米子处在$n$态。由于费米子满足泡利不相容原理,所以$
m \ne n$。

\begin{equation}\label{two fermions wave function}
\begin{gathered}
\psi _{m,n}^F (q_1 ,q_2 ) = \frac{1} {{\sqrt 2 }}\left|
{\begin{array}{*{20}c}
   {\psi _m (q_1 )} & {\psi _m (q_2 )}  \\
   {\psi _n (q_1 )} & {\psi _n (q_2 )}  \\
\end{array} } \right| \hfill \\
= \frac{1}{{\sqrt 2 }}\left( {\psi _m (q_1
)\psi _n (q_2 ) - \psi _m (q_2 )\psi _n (q_1 )} \right) \hfill \\
\end{gathered}
\end{equation}


现在我们湮灭一个粒子,这里就有两种选择了,假设我们湮灭一个$m$态的粒子,这样的一种操作用湮灭算符$a_m$表示,它的后果是只剩下一个粒子处在$n$态(相当于在行列式中拿掉一行)。记作:

\begin{equation}\label{ferminon anni}
a_m \left| {m,n} \right\rangle  \doteq \left| n \right\rangle
\end{equation}


我们还可定义产生算符,$a_m^{\dagger}$表示产生一个$m$态的粒子,对费米子系而言,有:


\begin{equation*}
a_m^\dag  \left| {m,n} \right\rangle  \doteq 0, a_m^\dag  \left| n
\right\rangle \doteq \left| {m,n} \right\rangle .
\end{equation*}


现在对只有一个$n$粒子的态,继续湮灭一个$m$粒子,由于系统中已经没有$m$粒子了,这样的过程是不存在的,记作:

\begin{equation*}
a_m \left| n \right\rangle \doteq 0 .
\end{equation*}


假设系统中有一个$m$粒子,则湮灭算符\index{Annihilation operator: 湮灭算符}$a_m$可以对这个量子态进行运算,计算结果是没有一个粒子,即所谓真空态($\left|
0 \right\rangle$),记作:

\begin{equation}\label{eq 3}
a_m \left| m \right\rangle \doteq \left| 0 \right\rangle
\end{equation}


按照这种约定,我们可以证明如下对易关系:

\begin{eqnarray}
% \nonumber to remove numbering (before each equation)
  \left\{ {a_m ,a_n^\dag  } \right\} &=& \delta _{m,n} \\
  \left\{
{a_m,a_n } \right\} &=& 0
\end{eqnarray}



对第一个对易式,对$m \ne n$我们可分别证明对量子态:$\left| {m,n}
\right\rangle$, $\left| m \right\rangle$, $\left| n \right\rangle$
和 $\left| 0 \right\rangle$都成立。比如对$\left| m
\right\rangle$证明:


\begin{center}
\begin{tabular}{|l|}
  \hline
  % after \\: \hline or \cline{col1-col2} \cline{col3-col4} ...
  $a_m a_n^\dag  \left| m \right\rangle  = a_m \left| {n,m}
\right\rangle  =  - a_m \left| {m,n} \right\rangle  =  - \left| n
\right\rangle $,\\
$a_n^\dag  a_m \left| m \right\rangle  = a_n^\dag
\left| 0 \right\rangle  = \left| n \right\rangle $,得证。 \\
  \hline
\end{tabular}
\end{center}


这里使用了费米子波函数的交换反对称性:

\begin{equation}
\left| {m,n} \right\rangle  = - \left| {n,m} \right\rangle ,
\end{equation}


再者:$a_m^\dag  \left| 0 \right\rangle  = \left| m \right\rangle
$,$a_n^\dag  \left| m \right\rangle  = \left| {n,m} \right\rangle
$;$a_n^\dag  \left| 0 \right\rangle  = \left| n \right\rangle
$,$a_m^\dag  \left| n \right\rangle  = \left| {m,n} \right\rangle
$;由此可证:$a_m^\dag a_n^\dag   =  - a_n^\dag  a_m^\dag $,
即:$\left\{ a_m^\dag , a_n^\dag \right\} = 0$。

对湮灭算符,$a_m \left| {m,n} \right\rangle  = \left| n
\right\rangle $,$a_n \left| {m,n} \right\rangle  =  - a_n \left|
{n,m} \right\rangle  =  - \left| m \right\rangle $;$a_n a_m \left|
{m,n} \right\rangle  = \left| 0 \right\rangle $,$a_m a_n \left|
{m,n} \right\rangle  =  - \left| 0 \right\rangle $;由此可证$\{a_m,
a_n\} =
0$。


注:这里的产生、湮灭算符想象为在行列式中增加或减少一行,行指标为量子态的指标。玻色系的证明思路大体相同,只是玻色系“行列式”的每一项都取正号。

按此思路将大大简化占有数表象下对易关系的证明,并有利于理解概念。


\subsubsection{玻色系统}

对玻色系统而言,假设两个玻色子\index{Boson: 玻色子},一个玻色子处在$m$态,另一个玻色子处在$n$态。如果$m \ne n$, 波函数应写为如下形式:

\begin{equation}\label{two bosons wavefunction}
\begin{gathered}
\psi _{m,n}^B (q_1 ,q_2 ) = \frac{1} {{\sqrt 2 }}\left|
{\begin{array}{*{20}c}
   {\psi _m (q_1 )} & {\psi _m (q_2 )}  \\
   {\psi _n (q_1 )} & {\psi _n (q_2 )}  \\
\end{array} } \right|_+ \hfill \\
= \frac{1}{{\sqrt 2 }}\left( {\psi _m (q_1 )\psi _n (q_2 ) + \psi _m
(q_2 )\psi _n (q_1 )} \right) \hfill \\
\end{gathered}
\end{equation}


如果$m=n$的话,$\psi_{m,m}^B(q_1,q_2) = \sqrt 2
\psi_m(q_1)\psi_m(q_2)$,但这样的波函数没有归一化,归一化后:

\begin{equation*}
\psi_{m,m}^B(q_1,q_2) = \psi_m(q_1) \psi_m(q_2) .
\end{equation*}

现在定义映射:

\begin{equation*}
\psi_m(q) \to a_m^{\dagger} \left| 0 \right\rangle, \psi_n(q) \to
a_n^{\dagger} \left| 0 \right\rangle
\end{equation*}


当$m \ne n$时,

\begin{center}

$\frac{1}{{\sqrt 2 }}\left( {\psi _m (q_1 )\psi _n (q_2 ) + \psi _m
(q_2 )\psi _n (q_1 )} \right) = \left| {m,n} \right\rangle  = \left|
{n,m} \right\rangle $

$\to a_m^\dag  a_n^\dag  \left| 0 \right\rangle = a_n^\dag  a_m^\dag
\left| 0 \right\rangle $.

\end{center}

$m=n$时,如果简单地把

\begin{center}

$\frac{1}{{\sqrt 2 }}\left( {\psi _m (q_1 )\psi _m (q_2 ) + \psi _m
(q_2 )\psi _m (q_1 )} \right) = \sqrt 2 \psi _m (q_1 )\psi _m (q_2
)$

$\to \left( {a_m^\dag  } \right)^2 \left| 0 \right\rangle$,

\end{center}

在形式上与$m \ne n$时相同,但$\left( {a_m^\dag  } \right)^2 \left| 0
\right\rangle$不是归一化的。归一化后的波函数即占有数表象\index{Occupation representation: 占有数表象}下的$\left|
{2_m } \right\rangle$:

\begin{equation}\label{two bosons}
\left| {2_m } \right\rangle  = \frac{1}{{\sqrt 2 }}\left( {a_m^\dag
} \right)^2 \left| 0 \right\rangle ,
\end{equation}


即:

\begin{equation*}
\left( {a^\dag  } \right)^2 \left| 0 \right\rangle  = a^\dag \left|
1 \right\rangle  = \sqrt 2 \left| 2 \right\rangle
\end{equation*}


推广到一般情形,考虑$N$个玻色子(写成“行列式”的形式):


\begin{equation}\label{N bosons}
\begin{gathered}
\psi_{k_1,k_2,...,k_N}^B(q_1,q_2,...,q_N) \hfill \\
= \left|
{\begin{array}{*{20}c}
   {\psi _{k_1 } (q_1 )} & {\psi _{k_1 } (q_2 )} & {...} & {\psi _{k_1 } (q_N )}  \\
   {\psi _{k_2 } (q_1 )} & {\psi _{k_2 } (q_2 )} & {...} & {\psi _{k_2 } (q_N )}  \\
   {...} & {...} & {...} & {...}  \\
   {\psi _{k_N } (q_1 )} & {\psi _{k_N } (q_2 )} & {...} & {\psi _{k_N } (q_N )}  \\
 \end{array} } \right|_ + \hfill \\
= \frac{1}{\sqrt {N!}} \sum\limits_P \psi_{k_1}(q_{P1})
\psi_{k_2}(q_{P2})...\psi_{k_N}(q_{PN}) \hfill \\
\end{gathered}
\end{equation}


$\sum\limits_P$是对位置的轮换, 体现了玻色子的交换对称性,
并不改变对态的占有数分布,
所以各轮换项都对应同一个占有数表象下的态矢量 $\left|n_1,n_2,...
\right\rangle$。这里的指标$k_1,k_2,...$(量子态)可能有重复,
假设有$n_1$个甲态, $n_2$个乙态等等, “行列式”形式的波函数(\ref{N
bosons})映射为$(a_1^{\dagger})^{n_1} (a_2^{\dagger})^{n_2}... \left|
0 \right\rangle$。它不是归一化的波函数, 归一化因子是$\frac{1}{{\sqrt
{\prod\limits_i {n_i !} } }}$,即:

\begin{equation}\label{Normalized bosons occ rep}
\left| {n_1 ,n_2 ,...} \right\rangle = \frac{1}{{\sqrt {
\prod\limits_i {n_i !} } }} (a_1^{\dagger})^{n_1}
(a_2^{\dagger})^{n_2} ... \left| 0 \right\rangle .
\end{equation}



\subsubsection*{证明如下}

\begin{center}

$\psi_{k_1,k_2,...,k_N}^B(q_1,q_2,...,q_N) = \frac{1}{\sqrt {N!}}
\sum\limits_P \psi_{k_1}(q_{P1})
\psi_{k_2}(q_{P2})...\psi_{k_N}(q_{PN})$

$\to \prod\limits_{i = 1}^N {a_{k_i }^\dag  \left| 0 \right\rangle
}$

\end{center}

如果所有的$k_i$相互都不同,那么波函数已经归一化了,若存在$k_i =
k_j$,波函数就不是归一化的。


考虑到玻色子允许相同量子态占据多个粒子,对所有的$\{k_i\}$进行重新分组,假设$k_1$态上占据了$n_1$个玻色子,等等。轮换$\prod\limits_{i
= 1}^N {\psi _{k_i } (q_{Pi} )}$中会出现很多相同的项,$\sum\limits_P
\psi ... \psi$中共有$N!$项,对占据$k_1$态的$n_1$个玻色子而言,$n_1
!$轮换对应相同的项。

所以:因多个玻色子可占据相同的态导致轮换$\sum\limits_P
\prod\limits_{i = 1}^N {\psi _{k_i } (q_{Pi} )}$中有$\prod\limits_i
{n_i !}$项是可以合并的,即:

\begin{equation*}
\frac{1}{{\sqrt {N!} }}\sum\limits_P {\psi ...\psi }  =
\frac{{\mathop \prod \limits_i n_i !}}{{\sqrt {N!} }}\sum\limits_{\{n_i \} } {\psi ...\psi }
\end{equation*}


合并后的求和就不是对轮换\index{Permutation: 轮换}$\sum\limits_P \psi ...
\psi$作的,而是对$n_i$分布作的,共有$\frac{{N!}}{{\prod\limits_i {n_i !} }}$项。因此:

\begin{equation*}
\sqrt{ \frac{\prod\limits_i {n_i !}}{N!} }  \sum\limits_{\{n_i \} } {\psi ...\psi } =  \left| n_1, n_2, ...\right\rangle
\end{equation*}

是归一的。于是:

\begin{equation*}
(a_1^{\dagger})^{n_1}(a_2^{\dagger})^{n_2}...\left| 0 \right\rangle
= \sqrt {\mathop \prod \limits_i n_i !} \sqrt {\frac{{\mathop \prod
\limits_i n_i !}}{{N!}}} \sum\limits_{\{ n_i \} } {\psi ...\psi }  =
\sqrt {\mathop \prod \limits_i n_i !} \left| {n_1 ,n_2 ,...}
\right\rangle .
\end{equation*}

公式(\ref{Normalized bosons occ rep})得证。


\subsubsection*{一个简单的例子:}

假设有3个玻色子,2个粒子在1态,1个粒子在2态。

\begin{equation*}
\left( {a_1^\dag  } \right)^2 a_2^\dag  \left| 0 \right\rangle  =
\frac{1} {{\sqrt {3!} }}\left| {...} \right|_{3 \times 3}  =
\frac{1} {{\sqrt {3!} }}\sum\limits_P {\psi \psi \psi }
\end{equation*}


$\sum\limits_P
{}$中共有六项。“3个玻色子,2个粒子在1态,1个粒子在2态。”组合数是:$\frac{3!}{2!1!}
= 3$, 即有2项是重复的。因此:

\begin{equation*}
\left( {a_1^\dag  } \right)^2 a_2^\dag  \left| 0 \right\rangle =
\frac{1} {{\sqrt {3!} }}\sum\limits_P {\psi \psi \psi }  =
\frac{{2!}} {{\sqrt {3!} }}\sum\limits_{\{ 2,1\} } {\psi \psi \psi }
\end{equation*}


$\sum\limits_{\{ 2,1\} } {\psi \psi \psi }$有3项, 分别是:

\begin{center}

$\psi_1(q_1)\psi_1(q_2)\psi_2(q_3)$,

$\psi_1(q_1)\psi_2(q_2)\psi_1(q_3)$,

$\psi_2(q_1)\psi_1(q_2)\psi_1(q_3)$.

\end{center}

$\frac{1}{\sqrt 3}\sum\limits_{\{ 2,1\} } {\psi \psi \psi }$
是归一化的,记为占有数表象下的态矢量:$\left| 2,1 \right\rangle$.
因此:

\begin{equation*}
\left| 2,1 \right\rangle = \frac{1}{\sqrt 2!} \left( {a_1^\dag  }
\right)^2 a_2^\dag  \left| 0 \right\rangle
\end{equation*}

\subsubsection{玻色子的对易关系:}

假设真空态已经是归一化的,$\left\langle 0 \right.\left| 0
\right\rangle  = 1$,$a^\dag  \left| 0 \right\rangle  = \left| 1
\right\rangle$,$a^\dag  \left| 1 \right\rangle  = \sqrt 2 \left| 2
\right\rangle$,$a^\dag  \left| 2 \right\rangle  = \frac{1}{{\sqrt 2
}}\left( {a^\dag  } \right)^3 \left| 0 \right\rangle  = \sqrt 3
\left| 3 \right\rangle $,...,推出:

\begin{equation}\label{boson recursion creat}
a^\dag  \left| n \right\rangle  = \sqrt {n + 1} \left| {n + 1}
\right\rangle .
\end{equation}


现在定义湮灭算符$a$,使得$a \left| 0 \right\rangle =
0$,假设量子态$\left| {m,k}
\right\rangle$,即一个玻色子处在$m$态,另一个玻色子处在$k$态,并且$m \ne k$。湮灭算符$a_m$,$a_k$应满足:

\begin{equation*}
a_m \left| {m,k} \right\rangle = \left| k \right\rangle , a_k \left|
{m,k} \right\rangle = \left| m \right\rangle
\end{equation*}


如果两个玻色子占据相同的态($m$):

\begin{equation*}
a_m (a_m^{\dag})^2 \left| 0 \right\rangle = a_m a_m^{\dag} \left|
{1_m} \right\rangle = \sqrt 2 a_m  \left| {2_m} \right\rangle = 2
a_m^{\dag} \left| 0 \right\rangle = 2 \left| {1_m} \right\rangle
\end{equation*}


湮灭算符的效果是拿掉“行列式”的一行;现在“行列式”的两行有相同指标,拿掉一行的方式有两种。

这意味着:$a \left| 2 \right\rangle = \sqrt 2 \left| 1 \right\rangle
$。这个结果可推广为:

\begin{equation*}
a \left| n \right\rangle = \sqrt n \left| {n-1} \right\rangle .
\end{equation*}

对$n$行有相同指标的行列式——相同的量子态上有$n$个粒子占据——
拿掉一行的方式有$n$种, 即:

\begin{equation*}
a (a^{\dag})^n \left| 0 \right\rangle = n (a^{\dag})^{n-1} \left| 0
\right\rangle , a \sqrt {n!} \left| n \right\rangle  = n \sqrt
{(n-1)!} \left| {n-1} \right\rangle
\end{equation*}

推出:

\begin{equation}\label{boson recursion minus}
a \left| n \right\rangle = \sqrt n \left| {n-1} \right\rangle
\end{equation}


由$a^\dagger$和$a$,可定义数算符(number operator)\index{Number operator: 数算符}:

\begin{equation}\label{Boson number operator}
a^{\dag} a \left| n \right\rangle = n \left| n \right\rangle
\end{equation}

和玻色对易关系:

\begin{equation}\label{Boson commutation}
[a, a^{\dagger}] =1 .
\end{equation}


\subsection{力学量的二次量子化形式}

定义单体算符:

\begin{equation}\label{one body operator}
A = \sum\limits_{m,n} {\left| m \right\rangle \left\langle m
\right|A\left| n \right\rangle \left\langle n \right|}  =
\sum\limits_{m,n} {A_{mn} \left| m \right\rangle \left\langle n
\right|}  \to A_{mn} a_m^\dag  a_n
\end{equation}

假设:$\left| {\nu \rho } \right\rangle  = a_\nu ^\dag  a_\rho ^\dag
\left| 0 \right\rangle $,那么:$\left\langle {\nu \rho } \right| =
\left\langle 0 \right|a_\rho  a_\nu $, 定义双体算符:

\begin{equation}\label{two body operator}
B = \sum\limits_{\lambda \mu ,\nu \rho } {\left| {\lambda \mu }
\right\rangle B_{\lambda \mu ,\nu \rho } \left\langle {\nu \rho }
\right|}  \to B_{\lambda \mu ,\nu \rho } a_\lambda ^\dag a_\mu ^\dag
a_\rho  a_\nu
\end{equation}

注意这里指标的次序, 如果是费米型算符, 指标的次序是不能任意颠倒的.

\subsection*{练习}

\begin{enumerate}

\item 

对玻色子, 请证明公式: 

\begin{eqnarray*}
a^\dagger \left| n \right\rangle & = & \sqrt{n+1} \left| n+1  \right\rangle  \\
a \left| n \right\rangle & = & \sqrt{n} \left| n-1 \right\rangle  \\
a^\dagger a \left| n \right\rangle & = & n \left| n \right\rangle \\
\left[ a, a^{\dagger} \right]  & = & 1 .\\
\end{eqnarray*}

\item 对玻色子,请证明:$[a^{\dag}a, a]=-a$, $[a^{\dag}a, a^{\dag}]=a^{\dag}$.

\end{enumerate}

\subsection*{阅读}

\begin{enumerate}
\item 

S. Doniach \& E.H. Sondheimer, Green's Functions For Solid State Physicists.

%\url{http://ishare.iask.sina.com.cn/f/12058139.html} (APPENDIX 1)

\end{enumerate}